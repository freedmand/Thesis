\subsection{Chord Progressions}

\textit{Chord progressions} are sequences of chords in a song. There is a degree of subjectivity in identifying progressions of chords as chords can be overlapping, involve notes that straddle and linger between chords, or can involve unknown or difficult to identify chord qualities. There are often perceptually obvious answers to what chords are playing based on musical cues ingrained in culture or common across songs.

Chord progressions are frequently notated with dashes in between, such as $Cmaj7 - Dm - G7 - Cmaj7$. Chord progressions are regarded as identical even if they are \textit{transposed}, or the pitch class of the root and bass notes of each chord is shifted a certain amount. For instance, $Dmaj7 - Em - A7 - Dmaj7$ describes an identical progression. A transposition-invariant representation is desired. Music theorists typically use roman numerals to represent the root and bass notes of the chord relative to the \textit{key} of the song, which outlines on which chord the song typically starts and ends.

The roman numeral $I$, or 1, corresponds to the start of the chord progression (in advanced music theory this is not always the case, but this is a simplification), and successive notes are represented by successive roman numerals. The chord progression $Cmaj7 - Dm - G7 - Cmaj7$ could be rewritten as $Imaj7 - ii - V7 - Imaj7$. It is important to note that \textit{minor} chords, or chords with qualities that are described as minor, are represented with lower case numerals, and the normal minor chord representation ($m$) can be omitted.