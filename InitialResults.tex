\chapter{Results}

\section{Variables Used and Notation} \label{varnot}

This section details a number of experiments run on the data. To compare chord progressions, only the Smith-Waterman algorithm ($SW$) is used, with normalization measures \textit{raw score} and ${SW}_{norm}$ being tested. Gap costs (${gap}_{open}$ and ${gap}_{ext}$) are allowed to vary from 0 through 128 (maximum range of an 8-bit integer). The Harte distance metric ($Harte$) and Tonal Pitch Space ($TPS$) are used to evaluate chord distances. Lastly, multiplication and subtraction factors $m_x$ and $m_s$ are used to round the chord distance metrics to integers with an expected value below 0. A full summary of the variables and their tested ranges is as follows:

\begin{align*}
\textbf{Variables}                       & \hspace{1cm} & \textbf{Values} \\
\text{Normalization } (norm)                    && \{\textit{raw score},{SW}_{norm}\} \\
\text{Gap open cost } ({gap}_{open})                    && [0-128] \\
\text{Gap extension cost } ({gap}_{ext})                && [0-128] \\
\text{Chord Distance Metric } (C_d)            && \{Harte, TPS\} \\
\text{Chord Distance Multiplier } (m_x)       && \{1, 30\} \\
\text{Chord Distance Subtraction Factor } (m_s) && \{0, 30\} \\
\end{align*}

\section{Fully connected pairwise corpus evaluation}

This is the primary experiment of this paper in which, for a given database $D$ of $n_d$ songs, the Smith-Waterman algorith is evaluated for every pairwise combination of two songs, resulting in \[ \frac{n_d (n_d - 1)}{2} \] comparisons (essentially the $n_d - 1$ triangle number). For each pairwise comparison, all 12 transpositions of one song relative to the other are calculated and the maximal comparison is chosen. For a visualization of the fully connected pairwise evaluation of the National Anthems dataset, see figure~\ref{fig:country_chord_graph}.