\chapter{Results}

\section{Variables Used and Notation}

This section details a number of experiments run on the data. To compare chord progressions, only the Smith-Waterman algorithm ($SW$) is used, with normalization measures \textit{raw score} and ${SW}_{norm}$ being tested. Gap costs (${gap}_{open}$ and ${gap}_{ext}$) are allowed to vary from 0 through 128 (maximum range of an 8-bit integer). The Harte distance metric ($Harte$) and Tonal Pitch Space ($TPS$) are used to evaluate chord distances. Lastly, multiplication and subtraction factors $m_x$ and $m_s$ are used to round the chord distance metrics to integers with an expected value below 0. A full summary of the variables and their tested ranges is as follows:

\begin{align*}
\textbf{Variables}                       & \hspace{1cm} & \textbf{Values} \\
\text{Normalization } (norm)                    && \{\textit{raw score},{SW}_{norm}\} \\
\text{Gap open cost } ({gap}_{open})                    && [0-128] \\
\text{Gap extension cost } ({gap}_{ext})                && [0-128] \\
\text{Chord Distance Metric } (C_d)            && \{Harte, TPS\} \\
\text{Chord Distance Multiplier } (m_x)       && \{1, 30\} \\
\text{Chord Distance Subtraction Factor } (m_s) && \{0, 30\} \\
\end{align*}

\section{Full connected pairwise computation}

\subsection{Clustering}

\subsection{Visualization}

\subsection{Ranking Fully Connected Pairwise Comparisons}

\subsection{Ranking Random N-Gram Search}

\subsection{Key-Finding Accuracy}

\section{Smith-Waterman Results}

\subsection{Normalization}

\item Problem, just given a raw score independent of song length or anything else

\item Attempt to solve: $normalization$ $equation$ $used$

\subsection{Clustering}

\item Tried out using graph clustering algorithm

\item Future: try k-means

\section{Visualization}

\subsection{Motivation}

\item Given over 100 pairwise (triangle) number comparisons between sets of songs, it is important to have a useful means of visualizing the data

\item Initially tried using Graphos and Circos and other graphing libraries

\item Demanded a level of interactivity unachievable with other programs

\subsection{Using D3.js}

D3, a Javascript library, if used to create the visualization. All songs are displayed as radial nodes around a circle, and pairwise connections as edges.

\item Using popular javascript graphing library, can construct a visualization based on hierarchical edge bundling, as used in other modern graphing libraries (curved edges using tension factors)

\item Can add interactivity using web server

\subsection{Using Twistd webserver}

\item Facile, easy use web server all in Python. Link C libraries to Python and have a rapid coding environment that has fast native C-code at its base

\item Can compare align results in a visual interface based on reconstructing the aligned data

\subsection{Audio playback}

\item Audio transposition problem, and solution using fewest localized transpositions

\item Following along using popcorn.js

\subsection{Public implementation}

\item Webserver available at chordmatch.com