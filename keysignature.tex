\subsection{Key Signature}

The \textit{key signature}, or just \textit{key}, of a song describes the \textit{tonic} or base harmony against which other chords perceptually resolve. The full mechanics are beyond the scope of an introductory music theory primer, but it is important to know that key signatures consist of a root note and a \textit{mode}---major or minor---which describe the mood of the piece and outline the expected chords and pitch classes used in the song.

The $C$ major key signature consists of all the white keys on the piano. The pitch classes of notes in a song in C major are expected to fall on these keys (there are exceptions but Western harmony typically plays under these rules). Only certain chords consist of pitch classes that are in $C$ major. Starting with the note $C$ and ascending upwards in triad chords the following list of harmonies is obtained: $(C, Dm, Em, F, G, Am, Bdim)$.

It can be useful to construct a template of acceptable pitch classes in C major. All the notes in a C major key signature starting at $C$ are $(C, D$, $E$, $F$, $G$, $A$, $B)$. In terms of intervals of each of these notes relative to the previous, starting at the second element ($D$), this list can be written $(+2, +2, +1, +2, +2, +2)$. With this template, it can be easy to obtain a list of acceptable pitch classes in other key signatures. For instance, the pitch classes $F\#$ major can be calculated by adding each interval to the last played note starting at $F\#$ obtaining $(F\#,G\#,A\#,B,C\#,D\#,F)$. The $A$ minor mode consists of all the white keys on the piano starting at $A$, and thus its template can be described $(+2,+1,+2,+2,+1,+2)$. See figure~\ref{fig:keysig} for a key signature template table.

\begin{figure}[h]
\centering
\begin{tabular}{lll}
\toprule
Key Signature Mode        & Interval Template (mod 12) \\
\midrule
Major       & +2, +2, +1, +2, +2, +2 \\
Minor       & +2, +1, +2, +2, +1, +2 \\
\bottomrule
\end{tabular}
\caption[Key Signature Mode Intervals]{The intervals from the previous note starting at the root that can be used to construct all the acceptable pitch classes in a key signature given the mode.}
\label{fig:keysig}
\end{figure}
