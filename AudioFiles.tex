\section{Audio Files}

\subsection{Storing Audio}

Music is stored digitally as a series of \textit{amplitudes}. Though beyond the technical scope of this paper, amplitudes represent the magnitude of compressions and rarefactions in the air that give sound to everything one can hear. Periodic fluctuations in amplitude represent the frequencies which give the listener a perceptual understanding of pitch. Digital storage of music invariably loses some information through \textit{quantization}, the process of making the continuous data sound waves represent into discrete values computers can comprehend. As another visual analogy, in order for a picture to be rendered on a computer screen it needs to fit into the rectangular shape of pixels (see figure~\ref{fig:rectangle})---likewise, audio needs to fit into discrete bins of amplitude values, losing some information.