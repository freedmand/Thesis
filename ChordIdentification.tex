\subsection{Classifying chords from audio files}

Chord identification from audio files is a difficult task that compounds the inexactness of pitch recognition and musical data collection into a more error-prone procedure. Chapter~\ref{computationchordextract} surveys existing techniques and their advantages and disadvantages, but this section will overview basic techniques used.

\textit{Machine learning} techniques are commonly used to classify chords from \textit{chroma features}. Chroma features represent the analyzed intensity of each pitch class by compounding frequencies from different octaves into a single bin. Machine learning involves training algorithms based on known data. In the context of music informatics, human-annotated or recognized chord progressions are referred to as \textit{ground-truth} sets\cite{BurgoyneEtAl_2011_AnExpeGrouSet}.