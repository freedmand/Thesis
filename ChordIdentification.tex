\subsection{Classifying chords from audio files}

Chord identification from audio files is a difficult task that compounds the inexactness of pitch recognition and musical data collection into a more error-prone procedure. Chapter~\ref{computationchordextract} surveys existing techniques and their advantages and disadvantages, but this section will overview basic techniques used.

\textit{Machine learning} algorithms are commonly used to classify chords from \textit{chroma features}. Chroma features represent the analyzed intensity of each pitch class by compounding frequencies from different octaves into a single bin. Machine learning involves training models based on known data and then observing how well they perform on new, or \textit{test}, data. In the context of music informatics, human-annotated or recognized chord progressions are referred to as \textit{ground-truth} sets\cite{BurgoyneEtAl_2011_AnExpeGrouSet}, so chord identification algorithms are trained and tested again ground-truth data.

Common statistical machine-learning procedures on which to train a model are \textit{Hidden Markov models} (HMM) and \textit{Bayesian networks}. Both are graphs in which nodes are connected and edges correspond to probabilities that a certain \textit{transition} will occur. The fine workings of these algorithms are beyond the scope of the paper, but it is important to know that HMMs and Bayesian networks involve probabilities that can be inferred through training data. These learned probabilities can be used on unknown, or test, data to classify chords by finding most likely paths through the data, called \textit{Viterbi paths}.