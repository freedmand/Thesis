\chapter{Methodology} \label{computationchordextract}

\section{Bridging the Gap: Subjective and Analyzed Notions of Harmony}

\subsection{The Divergence}

In chapter~\ref{chintro} notions of theory, human subjectivity, and analytical technique were shown to govern classification of information from musical songs; MIR attempts to reconcile the differences between these approaches in an automated fashion, but ground-truth data is difficult to find or even psychologically justify. This paper focuses on how analysis can be performed on music informatics retrieval (MIR) tasks using computationally analyzed data compared with human-derived ground-truth data, with a focus on chord progressions. Subjective perception is important as music only exists through the ears of a listener who often is informed and subconsciously guided by cultural cues; analytical computation and statistical methods also exhibit strong explanatory power and are shown to yield novel insights. 

% "chord labels form an abstraction of musical content with substantial explanatory power" \cite{de2008tonal}.

\subsection{Harmony}

One question

\subsection{Background}

Chord progressions

\item Using state-of-the-art chord progression extraction algorithms, how can we feabibly compare two harmonic progressions inexactly?

\subsection{Methodology}

\item How common algorithms work

\subsection{MIR Contest}

\item Evaluation
\item Khalkevid and other chord extraction methods

\subsection{Chordino}

\item Vamp plugin
\item Good recall
\item Got it working as a command line interface with Python and C
\item Acknowledge common pitfalls in identification