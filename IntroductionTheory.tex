% \section{Introduction to Harmony and Music Theory}

% \subsection{Basic Building Blocks of Musical Harmony}

% A \textit{note} is the most basic semantic element in music, consisting of a duration and a pitch. A pitch is commonly misunderstood to be a spectral frequency, in the pure physical definition of the term; rather, a pitch involves a perceived human element. While \textit{pitch} and \textit{frequency} are often used interchangeably, they are invariably distinct. Notes on the piano compose. Theorist Ernst Terhardt, who pioneered the term \textit{virtual pitch}, which is outside the scope of this paper but describes perceived pitches whose corresponding frequencies are not present, uses the following visual analogy to present the concept of something perceived but not actually present:

% A note consists of a duration and \textit{pitch}. A pitch corresponds to a perceived frequency of sound and differs from a purely physical definition

% A work of music consists in simplest terms of a collection of notes. A \textit{note} is comprised of


% The basic building block of music is the \textit{note}. A note consists of a \textit{pitch} and a duration. A pitch corresponds to the human perception of the \textit{note}

% In western music theory, the basic building block of all songs is the \textit{note}. A note has a given base frequency, or pitch, and duration. A chord is a combination of notes