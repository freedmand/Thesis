\subsection{Classifying pitches from audio files}

The \textit{Fourier transform} is a mathematical algorithm that can be used
to extract frequencies from a series of amplitudes. Most modern audio files
are sampled at 44,000 $Hz$, which means that there
are 44,000 data points for every second of audio. Let $sr$ denote the sampling
rate of an audio file in $Hz$. For a given segment of audio consisting of
$n$ data points, the Fourier transform returns $n$ values, where the magnitude
of the $i$th value corresponds to the strength of the frequency $\frac{sr
   \cdot i}{n}Hz$. A graph of these values with time along the x-axis, frequency
along the y-axis, and intensity represented by color is called a \textit{spectrogram}.
An example spectrogram of the Beatles song \textit{Eleanor Rigby} is given
in figure~\ref{fig:eleanor_rigby}.