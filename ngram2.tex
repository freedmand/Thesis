To attempt to enhance the power of localized comparisons, \textit{n-grams} can be used to measure distances between chord subsequences. An n-gram is just a sequence of a fixed length $n$, where 2-gram would refer to a sequence of length 2, 3-gram would refer to a sequence of length 3, and so on. Using two sliding windows fixed at length $n_g$, both songs can be traversed to calculate chordal distances between every n-gram in both songs and sum the distances. The chordal distance between two n-grams can be calculated by summing the chordal distance of each vertical pair of chords within the n-gram (see figure~\ref{fig:ngram}), essentially the global comparison algorithm of equal length applied within a small window. This can be expressed: \[\sum_{i=0}^{n1 - n_g} \sum_{j=0}^{n2 - n_g} \left( \sum_{k=0}^{n_g} C_d({c1}_{i+k}, {c2}_{j+k}) \right) \]

To take into account songs in different keys, transpositions can easily be factored in to find the key with the maximal score: \[\max_{s=0}^{12} \sum_{i=0}^{n1 - n_g} \sum_{j=0}^{n2 - n_g} \left( \sum_{k=0}^{n_g} C_d({c1}_{i+k}, t_s({c2}_{j+k})) \right) \]

A nice advantage of using n-grams is that all local regions of similarity are included, even if it's the last $n_g$ chords of one song and the first $n_g$ of another. A disadvantage is the running time of the algorithm, which is quadratic and in practice proved to be slower than other measures. Another disadvantage is that the window size, $n_g$, must be decided ahead of time and thus limits the flexibility of the algorithm. Finally, there is still not a good means of compensating for small errors, like scattered occurrences of erroneous chords.

I implemented a simple Python program that calculates this algorithm using 4-grams and the Harte chord distance metric (see section CITE) to show pairwise comparisons between extracted chord progressions of the songs \textit{Let It Be} by The Beatles, \textit{When I Come Around} by Greenday, and a live rendition of \textit{Let It Be}. All these songs have the same base chord progression, but I expected \textit{Let It Be} and its live rendition to have the maximal similarity. For the maximal key transposition, I collected every pair of 4-grams between the songs, placed the distances in bins, and plotted as a histogram. The maximum distance between any two 4-grams is 4.0, and I chose a bin size of 0.2. On the graphic I produced, the x-axis corresponds to the 4-gram distance bins, and the y-axis corresponds to frequency of occurrence of all the n-grams that were placed in the bins in the pairwise comparison between the two songs. I also list the average 4-gram distance and the cumulative frequency of 4-gram distances greater than or equal to 4 to see the frequency of more similarly matched 4-grams. The transposition in semitones is also listed. I manually checked the key of each song and found that both \textit{Let It Be} renditions to be in the key of $C$ major and \textit{When I Come Around} to be in $F\#$ major, corresponding accurately to a transposition of $6$ semitones. The results are shown in figure~\ref{fig:ngramdist}.