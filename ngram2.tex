
To take into account songs in different keys, transpositions can easily be factored in to find the key with the maximal score: \[\max_{s=0}^{12} \sum_{i=0}^{n1 - n_g} \sum_{j=0}^{n2 - n_g} \left( \sum_{k=0}^{n_g} C_d({c1}_{i+k}, t_s({c2}_{j+k})) \right) \]

A nice advantage of using n-grams is that all local regions of similarity are included, even if it's the last $n_g$ chords of one song and the first $n$ of another. A disadvantage is the running time of the algorithm, which is quadratic and in practice proved to be slower than other measures. Another disadvantage is that the window size, $n_g$, must be decided ahead of time and is limited. Finally, there is still not a good means of compensating for small errors, like scattered occurrences of erroneous chords.

I implemented a simple Python program that calculates this algorithm using 4-grams and the Harte chord-wise distance metric (see section CITE) to show pairwise comparisons between the songs \textit{Let It Be} by The Beatles, \textit{When I Come Around} by Greenday, and a live rendition of \textit{Let It Be}. For the maximal key transposition, every pair of 4-grams' distances are collected and plotted as a histogram. The maximum distance between any two 4-grams is 4.0. The x-axis corresponds to the 4-gram distance and the y-axis corresponds to frequency of occurrence in the pairwise comparison between the two songs. The results are shown in figure~\ref{fig:ngramhist}.