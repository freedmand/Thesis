\section{Distance Function between Chords and Gap Costs}

\subsection{Considerations}

\item Should perform well comparing extracted data set to ground truth sets following MIR data (also separate from MIR set is valuable as untrained data).

\item Should classify chords with similar notes as similar, and place emphasis moreso on pitch class so as to avoid errors with key recognition and such

\subsection{Tonal Pitch-Step Distance}

\item Motivations, research, circle-of-fifths step

\item Dependence on key information, impracticality

\subsection{Harte Distance Metric}

The Harte chord distance metric, proposed by Christopher Harte in \cite{harte2010towards}, is a simple and convenient means of calculating a score representing the distance between any two chords.

Letting $P_c$ represent the given pitch classes of a chord, we can define Harte's formula as follows:
\[ H(c_1,c_2) = \frac{\left| P_c(c_1) \cap P_c(c_2) \right|}{\left| P_c(c_1) \cup P_c(c_2) \right|} \]

\item Advantages: easy to implement, fast, concise

\item Disadvantages: does not consider full range of key in music. Ignores root.

\subsection{Other chord distance metrics}

\item List others in Survey paper. Outside scope but could be tested in the future.