
\item Using a sliding window of n-grams (tried 4), can we see the distribution of top scores and make some sort of claim based on that.

\item Advantages: Compares localized alignments, that is alignments in different positions.

\item Disadvantages: Computationally expensive. Does not work for arbitrary window lengths without another exponent in computation time required. Does not account for marginal errors well.

\subsection{Minimum Edit Distance}

\item Global comparison, assign cost to different edit distances and find minimum required "transformation" between two songs. Uses dynamic programming and can be done in quadratic time.

\item Disadvantages: global alignments only.

\subsection{Smith-Waterman}

Sequence alignment refers to the computational task of trying to find common subsequences within two different sequences with minimal gaps. As a simplified example, consider trying to align the following words:

% -atte
%  || |
% pat-e

\section{Smith-Waterman Algorithm}

The Smith-Waterman algorithm is used to find optimal local alignments between two sequences based on a cost matrix for symbols of the alphabet being used and defined gap costs.

A matrix $H$ is constructed in the following manner:
\[ H(i,0) = 0, 0 \leq i \leq m \]
\[ H(0,j) = 0, 0 \leq j \leq n \]
\[ H(i,j) = \hbox{max} \begin{cases} 0 \\ H(i-1,j-1)+s(a_i,b_j) \\ \hbox{max}_{k \geq 1} \left\{ H(i-k,j)+W_k \right\} \\ \hbox{max}_{l \geq 1}\left\{ H(i, j-l)+W_l) \right\} \end{cases}  \]

\item Localized comparison, dynamic programming, find minimum number of required "transformations" and optimal localized slice. Dynamic programming

\item Works well with inexact data, can deal with common pitfalls of chord extraction.

\item Isolates similar substructures.

\item Does not penalize closely related chords, i.e. Em7 and Gmaj

\item Difficulty in extracting multiple "best" options but good at finding one top contender

\item Difficulty in comparing scores

\subsection{Example Smith-Waterman Algorithm}

The following example shows how the Smith-Waterman algorithm could be applied to the alphabet of musical chord symbols:

\[ \begin{array}{cccccccc} & - & Cmaj & Fmaj & Cmaj & Gmaj & Fmaj & Cmaj & \\ - & 0 & 0 & 0 & 0 & 0 & 0 & 0 & \\ Cmaj & 0 & 3 & 0 & 3 & 0 & 0 & 3 & \\ Fmaj & 0 & 0 & 6 & 2 & 1 & 3 & 0 & \\ Dmin & 0 & 0 & 2 & 4 & 0 & 0 & 1 & \\ Gmaj & 0 & 0 & 0 & 0 & 7 & 3 & 0 & \\ Cmaj & 0 & 3 & 0 & 3 & 3 & 5 & 6 & \\ Gmaj & 0 & 0 & 1 & 0 & 6 & 2 & 3 & \\ Gmaj & 0 & 0 & 0 & 0 & 3 & 4 & 0 & \\ \end{array} \]

This example assumes the following costs:

If the chord begins on the same root, then add 3 \\
Otherwise, subtract 4 \\