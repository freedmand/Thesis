Western music is based on a division of 12 distinct frequencies per \textit{octave}. An octave is an \textit{interval}, or distance between two frequencies, that corresponds to a power of 2 multiplication. Musical pitch is perceived in a logarithmic scale---one octave above a given perceived frequency is double that frequency; one octave below is half that frequency. The progression of notes containing all 12 pitches in succession in an octave is called a \textit{chromatic scale}. A \textit{semitone}, or \textit{half-step}, is the smallest interval, equal to $1/12$ of an octave. $n$ semitones above a given frequency $f_0$ or $-n$ below can be calculated as $f_0 \cdot 2^{n/12}$.

\textit{Note names} are used to classify the pitches in the chromatic scale. Note names consist of a base name and 0 or more \textit{accidentals}. The base names of a note correspond to the white keys on a piano---in any one given octave there are the following names: $C$, $D$, $E$, $F$, $G$, $A$, and $B$. A base note name can optionally be decorated with an indefinite number of sharps ($\#$) or flats ($b$), but not both, in the note name. This can be illustrated with the context-free grammar in figure ~\ref{fig:cfgnote}.

\begin{figure}[h!]
\begin{center}
\begin{align}
NoteName &\to BaseNote \mid BaseNote\ SharpAccidentals \mid BaseNote\ FlatAccidentals \\
BaseNote &\to \mathbf{C} \mid \mathbf{D} \mid \mathbf{E} \mid \mathbf{F} \mid \mathbf{G} \mid \mathbf{A} \mid \mathbf{B} \\
SharpAccidentals &\to \mathbf{\#}\ SharpAccidentals \mid \mathbf{\#} \\
FlatAccidentals &\to \mathbf{b}\ FlatAccidentals \mid \mathbf{b}
\end{align}
\caption{Context-free grammar of a note name}
\label{fig:cfgnote}
\end{center}
\end{figure}


Sharps and flats are referred to as accidentals. Each additional ($\#$) increases the pitch to which the note name refers by 1 semitone; likewise, each ($b$) decreases the pitch by 1 semitone.  The black keys on the piano represent pitches 1 semitone in between the surrounding white keys. Each white key is either 1 semitone or 2 semitones apart, depending on if a black key is in the middle. For instance, $C$ and $D$ are 2 semitones apart, whereas $E$ and $F$ are 1 semitone apart. See figure ~\ref{fig:piano}.
