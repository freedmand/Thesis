Automatic pitch identification of audio is inexact. The peak-finding approach only captures salient frequency values, which do not necessarily imply the corresponding pitches are present. The complex, rich sound of instruments and voices have \textit{overtones}, or frequencies that are multiples of the perceived pitch, presenting obscure samples. Noise and extraneous sound clutter recordings. The quantization of audio files and impreciseness of recording equipment and synthesizers prevents perfect data collection. Multiple melodic parts can make it hard to isolate regions or discern between instrumental lines. There are more complicated means of identifying pitches that take into account \textit{timbral} aspects of instruments, color and overtones, but the subjective nature of pitch interpretation means definitive truths are hard to establish, and the computational task of hearing as a human might broaches the field of artificial intelligence.