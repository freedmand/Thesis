The calculation of $j$ consists of deriving four sets of tonal space for each chord Lerdahl outlines in which each subsequent level is a superset of the previous level. Level 1, $L_1$, of a chord $c$ returns a set only containing the chord's root note: \[L_1(c) = \{root(c)\}\] Level 2, $L_2$, returns a set containing a given chord's root note and fifth interval: \[L_2(c) = \{root(c), V(c)\}\] Level 3, $L_3$, returns a set containing all the pitch classes in a given chord: \[L_3(c) = P_c(c)\] Finally, level 4, $L_4$, consists of all the pitch classes in the key signature given: \[L_4(c,k) = P_c(k)\] With each level, one set from each chord being compared is used and the cardinality of the \textit{symmetric difference} ($\triangle$) is computed. The symmetric difference corresponds to the size of the set of unique elements between both sets being compared. For instance, $\{1,2,3\} \triangle \{2,3,4\} = \{1,4\}$. $j$ is calculated as follows:

\[ j(c_1,c_2,k) = \frac{|L_1(c_1) \triangle L_1(c_2)| + |L_2(c_1) \triangle L_2(c_2)| + |L_3(c_1) \triangle L_3(c_2)| + |L_4(c_1,l) \triangle L_4(c_2,l)|}{2} \]

\subsubsection{Key Finding Using Tonal Pitch Step}
