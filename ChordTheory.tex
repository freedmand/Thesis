\subsection{Chords and Harmonies}

A chord consists of a combination of notes sounding simultaneously or close enough in succession to resemble a texture. The \textit{Harvard Dictionary of Music} defines a chord as consisting of at least three notes\cite{harvdict}. A chord perceptually describes the notes that are contained within.

Chords are commonly labeled with qualities, which describe the intervals between the pitch classes involved, invariant of octave. Notes can be replicated across octaves as long as they occur at least once, and the ordering can be changed. Different orderings and octave choices in a chord are called \textit{voicings}.

A \textit{major} chord consists of a \textit{root note}, the base pitch class from which successive intervals are constructed, and pitch classes 4 semitones and 7 semitones above the root note modulus 12. This can be notated as a list of intervals, $0, +4, +7$ mod 12, but for convenience the root note corresponding to interval $0$ can be omitted. See figure~\ref{fig:qualitytable} for a sample of commonly named chord qualities and the associated intervals. \\

\begin{figure}[h!]
\centering
\begin{tabular}{lll}
\toprule
Chord Quality       & Shorthand & Intervals from Root (mod 12) \\
\midrule
Major               &           & +4, +7     \\
Major 6th           & 6         & +4, +7, +8 \\
Major 7th           & maj7      & +4, +7, +11\\
Minor               & m         & +3, +7     \\
Minor 6th           & m6        & +3, +7, +9 \\
Minor 7th           & m7        & +3, +7, +10\\
Dominant 7th        & 7         & +4, +7, +10\\
Augmented           & aug       & +4, +8     \\
Diminished          & dim       & +3, +6     \\
Diminished 7th      & dim7      & +3, +6, +9 \\
Half-diminished 7th & m7b5      & +3, +6, +10\\
\bottomrule
\end{tabular}
\caption{Common chord qualities and associated intervals}
\label{fig:qualitytable}
\end{figure}

Chords are labeled with their root note followed by their quality, like $Eb$ minor, $B$ augmented, or $F$ half-diminished 7th. A chord with only 3 notes in which successive intervals are within an octave from the root note is called a \textit{triad}. A $C$ major triad is demonstrated in figure~\ref{fig:cmajorchord1}.
% The octave of these successive notes can be incremented or decremented freely, and the chord is still considered to have a \textit{major} quality. See figure for a C major chord in root position.

% Chords feature a \textit{root note} which is the note around which the \textit{quality} is based.

% The lowest note (note with the lowest pitch) within a \textit{root position} chord is called its \textit{root note}.

% A chord typically has a root note, the note with the lowest pitch, and a quality, which describes the arrangement of successive notes' pitches.