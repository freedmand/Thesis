\subsection{Chords and Harmonies}

A chord consists of a combination of notes sounding simultaneously or close enough in succession to resemble a texture. The \textit{Harvard Dictionary of Music} defines a chord as consisting of at least three notes\cite{harvdict}. A chord perceptually describes the notes that are contained within.

Chords feature a \textit{root note} which is the note around which the \textit{quality} is based.

The lowest note (note with the lowest pitch) within a \textit{root position} chord is called its \textit{root note}.

A chord typically has a root note, the note with the lowest pitch, and a quality, which describes the arrangement of successive notes' pitches.