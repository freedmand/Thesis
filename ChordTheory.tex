\subsection{Chords and Harmonies}

A chord consists of a combination of notes sounding simultaneously or close enough in succession to resemble a texture. The \textit{Harvard Dictionary of Music} defines a chord as consisting of at least three notes\cite{harvdict}. A chord perceptually describes the notes that are contained within.

Chords are commonly labeled with qualities, which describe the intervals between the pitches involved. A \textit{root note} describes the base upon which successive intervals are based. For example, given a specified root, a \textit{major} chord consists of notes 3 semitones above the root and 6 semitones above the root, invariant of octave. This means that notes in a major chord must be 3 and 6 semitones above the root modulus 12. See figure~\ref{fig:qualitytable} for a sample of commonly named chord qualities and the associated intervals.

\begin{figure}[h!]
\begin{center}
\begin{tabular}{lll}
\toprule
Chord Quality       & Shorthand & Intervals from Root (mod 12) \\
\midrule
Major               &           & +3, +6     \\
Major 6th           & 6         & +3, +6, +8 \\
Major 7th           & maj7      & +3, +6, +10\\
Minor               & m         & +2, +6     \\
Minor 6th           & m6        & +2, +6, +8 \\
Minor 7th           & m7        & +2, +6, +9 \\
Dominant 7th        & 7         & +3, +6, +9 \\
Augmented           & aug       & +3, +7     \\
Diminished          & dim       & +2, +5     \\
Diminished 7th      & dim7      & +2, +5, +8 \\
Half-diminished 7th & m7b5      & +2, +5, +9 \\
\bottomrule
\end{tabular}
\caption{Common chord qualities and associated intervals}
\label{fig:qualitytable}
\end{center}
\end{figure}

A chord with only 3 notes in which successive intervals are in the same octave is called a \textit{triad}. A C major triad is demonstrated in figure~\ref{fig:cmajorchord1}.
% The octave of these successive notes can be incremented or decremented freely, and the chord is still considered to have a \textit{major} quality. See figure for a C major chord in root position.

% Chords feature a \textit{root note} which is the note around which the \textit{quality} is based.

% The lowest note (note with the lowest pitch) within a \textit{root position} chord is called its \textit{root note}.

% A chord typically has a root note, the note with the lowest pitch, and a quality, which describes the arrangement of successive notes' pitches.