\chapter{A Brief Overview of Music}

\section{A Primer on Western Music Theory}

\subsection{Notes: The Basic Building Block}

In music, a \textit{note} is the most basic element. A note is based on pitch, a subjective and perceptual property. Though the pitch of a note is closely related and usually resembles its objective physical frequency (as measured in Hertz, or cycles per second, of a waveform), pitch differs in that its semantic meaning is derived from the listener. A note also consists of a duration.

Western music is based on a division of 12 distinct frequencies per \textit{octave}. An octave is an \textit{interval}, or distance between two frequencies, that corresponds to a power of 2 multiplication. Musical pitch is perceived in a logarithmic scale---one octave above a given perceived frequency is double that frequency; one octave below is half that frequency. The progression of notes containing all 12 pitches in succession in an octave is called a \textit{chromatic scale}. A \textit{semitone}, or \textit{half-step}, is the smallest interval, equal to $1/12$ of an octave. $n$ semitones above a given frequency $f_0$ or $-n$ below can be calculated as $f_0 \cdot 2^{n/12}$.

\textit{Note names} are used to classify the pitches in the chromatic scale. Note names consist of a base name and 0 or more \textit{accidentals}. The base names of a note correspond to the white keys on a piano---in any one given octave there are the following names: $C$, $D$, $E$, $F$, $G$, $A$, and $B$. A base note name can optionally be decorated with an indefinite number of sharps ($\#$) or flats ($b$), but not both in the same note name. Each additional $\#$ increases the pitch to which the note name refers by 1 semitone; likewise, each $b$ decreases the pitch by 1 semitone. The black keys on the piano represent pitches 1 semitone in between the surrounding white keys. Each white key is either 1 semitone or 2 semitones apart, depending on if a black key is in the middle. For instance, $C$ and $D$ are 2 semitones apart, whereas $E$ and $F$ are 1 semitone apart. See figure $\ref{fig:piano}$.

The divergence between exactly measurable and calculable to subjectively derived from perception---at the most basic element of music---gives birth to the field of \textit{music informatics retrieval} (MIR), devoted to automatically extracting data and classifying features from works of music.