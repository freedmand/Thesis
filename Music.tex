\chapter{A Brief Overview of Music}

\section{A Primer on Western Music Theory}

\subsection{Notes: The Basic Building Block}

In music, a \textit{note} is the most basic element. A note is based on pitch, a subjective and perceptual property. Though the pitch of a note is closely related and usually resembles its objective physical frequency (as measured in Hertz, or cycles per second, of a waveform), pitch differs in that its semantic meaning is derived from the listener. A note also consists of a duration.

Western music is based on a division of 12 distinct frequencies per \textit{octave}. An octave is an \textit{interval}, or distance between two frequencies, that corresponds to a power of 2 multiplication. Musical pitch is perceived in a logarithmic scale---one octave above a given perceived frequency is double that frequency; one octave below is half that frequency. A \textit{semitone} is the smallest interval, equal to $1/12$ of an octave. $n$ semitones above a given frequency $f_0$ can be calculated as $f_0 \cdot 2^{n/12}$.

\textit{Note names} are used to classify these frequencies, irrespective of octave. Note names correspond to the white keys on a piano---in any one given octave there are the base note names $C$, $D$, $E$, $F$, $G$, $A$, and $B$ (see figure 2.1). Each of these base note names can be decorated with an indefinite number of sharps ($\#$) and flats ($b$)

The divergence between exactly measurable and calculable to subjectively derived from perception---at the most basic element of music---gives birth to the field of \textit{music informatics retrieval} (MIR), devoted to automatically extracting data and classifying features from works of music.