\chapter{Code Implementation}

\section{Smith-Waterman}

\subsection{Using the chord alphabet}

\item Chord data structures (alphabet), base 12 representation

\item Ignoring duration (future work)

\subsection{Bitwise representation of harmonies}

\item Cleverness with representing chords and harmonies as integers and allowing simple bitwise operations and bitmasks

\subsection{MIPS Implementation}

\item Used BU implementation, extreme speed results, but only for DNA and Protein sequences, the usual use case of Smith-Waterman

\item Optimized implementation for use with chord alphabet

\item Initialize distance matrix, extremely fast

\subsection{Speed considerations}

\item Initially implemented in Python, experience 100,000x speed-up

\subsection{Future considerations}

\item Fast SW extremely useful but also highly restrictive in its compactness and efficiency -- hard to modify

\item Adapt using local transpositions