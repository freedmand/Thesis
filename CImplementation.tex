\chapter{Code Implementation}

\section{Smith-Waterman}

\subsection{Using the chord alphabet}

The Smith-Waterman algorithm is typically used in bioinformatic applications in which the alphabet is typically restricted to DNA or protein characters. To use an alphabet that contains all the chord symbols, a bijective function can be established between every type of chord and a unique 16 bit integer. Recall a chord can be described with the following grammar:

\begin{align*}
Chord &\to Root\ Harmony \ Bass \mid \textbf{NoChord} \\
Root &\to PitchClass \\
Bass &\to PitchClass \\
PitchClass &\to \textbf{A} \mid \textbf{A#/Bb} \mid \textbf{B} \mid \textbf{C} \mid \textbf{C#/Db} \mid \textbf{D} \mid \textbf{D#/Eb} \mid \textbf{E} \mid \textbf{F} \mid \textbf{F#/Gb} \mid \textbf{G} \mid \textbf{G#/Ab} \\
Harmony &\to \textbf{maj} \mid \textbf{6} \mid \textbf{maj7} \mid \textbf{m} \mid \textbf{m6} \mid \textbf{m7} \mid \textbf{7} \mid \textbf{aug} \mid \textbf{dim} \mid \textbf{dim7} \mid \textbf{m7b5} \mid \textbf{UnknownHarmony}
\end{align*}

\item Chord data structures (alphabet), base 12 representation

\item Ignoring duration (future work)

\subsection{Bitwise representation of harmonies}

\item Cleverness with representing chords and harmonies as integers and allowing simple bitwise operations and bitmasks

\subsection{MIPS Implementation}

\item Used BU implementation, extreme speed results, but only for DNA and Protein sequences, the usual use case of Smith-Waterman

\item Optimized implementation for use with chord alphabet

\item Initialize distance matrix, extremely fast

\subsection{Speed considerations}

\item Initially implemented in Python, experience 100,000x speed-up

\subsection{Future considerations}

\item Fast SW extremely useful but also highly restrictive in its compactness and efficiency -- hard to modify

\item Adapt using local transpositions