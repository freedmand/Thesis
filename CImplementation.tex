\chapter{Code Implementation}

\section{Smith-Waterman}

\subsection{Using the chord alphabet}

The Smith-Waterman algorithm is typically used in bioinformatic applications in which the alphabet is typically restricted to DNA or protein characters. To use an alphabet that contains all the chord symbols, a bijective function can be established between every type of chord and a unique 16 bit integer. Recall a chord can be described with the following grammar:

\begin{align*}
Chord &\to Root\ Harmony \ Bass \mid \textbf{NoChord} \\
Root &\to PitchClass \\
Bass &\to PitchClass \\
PitchClass &\to \textbf{A} \mid \textbf{A#/Bb} \mid \textbf{B} \mid \textbf{C} \mid \textbf{C#/Db} \mid \textbf{D} \mid \textbf{D#/Eb} \mid \textbf{E} \mid \textbf{F} \mid \textbf{F#/Gb} \mid \textbf{G} \mid \textbf{G#/Ab} \\
Harmony &\to \textbf{maj} \mid \textbf{6} \mid \textbf{maj7} \mid \textbf{m} \mid \textbf{m6} \mid \textbf{m7} \mid \textbf{7} \mid \textbf{aug} \mid \textbf{dim} \mid \textbf{dim7} \mid \textbf{m7b5} \mid \textbf{UnknownHarmony}
\end{align*}

Notice that $|Root| = |Bass| = |PitchClass| = 12$ and $|Harmony| = 12$. A bijective $p$ between $PitchClass$ and integer from 0 through 11 can be established as follows:

\begin{tabular}{ll}
\toprule
$PitchClass$ & $p(PitchClass)$ \\
\midrule
\textbf{A} & 0 \\
\textbf{A#/Bb} & 1 \\
\textbf{B} & 2 \\
\textbf{C} & 3 \\
\textbf{C#/Db} & 4 \\
\textbf{D} & 5 \\
\textbf{D#/Eb} & 6 \\
\textbf{E} & 7 \\
\textbf{F} & 8 \\
\textbf{F#/Gb} & 9 \\
\textbf{G} & 10 \\
\textbf{G#/Ab} & 11 \\
\bottomrule
\end{tabular}

Likewise, a bijection function $h$ between $Harmony$ and an integer from 0 through 11 can be established as follows:

\begin{tabular}{ll}
\toprule
$Harmony$ & $h(Harmony)$ \\
\midrule

\textbf{maj} & 0 \\
\textbf{6} & 1 \\
\textbf{maj7} & 2 \\
\textbf{m} & 3 \\
\textbf{m6} & 4 \\
\textbf{m7} & 5 \\
\textbf{7} & 6 \\
\textbf{aug} & 7 \\
\textbf{dim} & 8 \\
\textbf{dim7} & 9 \\
\textbf{m7b5} & 10 \\
\textbf{UnknownHarmony} & 11 \\
\bottomrule
\end{tabular}

In base 12, a chord can be represented as integer.

\item Chord data structures (alphabet), base 12 representation

\item Ignoring duration (future work)

\subsection{Bitwise representation of harmonies}

\item Cleverness with representing chords and harmonies as integers and allowing simple bitwise operations and bitmasks

\subsection{MIPS Implementation}

\item Used BU implementation, extreme speed results, but only for DNA and Protein sequences, the usual use case of Smith-Waterman

\item Optimized implementation for use with chord alphabet

\item Initialize distance matrix, extremely fast

\subsection{Speed considerations}

\item Initially implemented in Python, experience 100,000x speed-up

\subsection{Future considerations}

\item Fast SW extremely useful but also highly restrictive in its compactness and efficiency -- hard to modify

\item Adapt using local transpositions