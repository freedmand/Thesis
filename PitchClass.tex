\subsection{Pitch Class}

Though the chromatic scale contains 12 notes, there are an infinite amount
of ways to represent any singular pitch within the octave. For instance,
a $C\sharp$ pitch can be represented as a $D\flat$, a $B\sharp\sharp$ (``B-double-sharp''),
or a $F\flat\flat\flat\flat$, among other possibilities. There are notational
reasons to represent a pitch in these ways; outside of \textit{equal temperament},
the tuning system upon which Western music is based, these notes sound different
and have different perceptual frequencies. In equal temperament, which is
an assumption guiding this paper, all these different representations of
the same chromatic note have identical pitches.

A \textit{pitch class} is the collection of all identical pitches across
all octaves. The $C\sharp$ pitch class, for instance, contains all the $C\sharp$
pitches over all the octaves, the $D\flat$ pitches over all the octaves,
and any other pitch that represents the same chromatic note across all octaves.